\documentclass[
    a4paper,
    12pt,
    %twocolumn,
    headings=normal
]{scrreprt}
    \usepackage{url}
    \usepackage[margin=1.0in]{geometry}
    \usepackage{amssymb}
    \usepackage{enumerate}
    \usepackage{enumitem}
    \usepackage{csquotes}
    \usepackage{graphicx}
    \usepackage{xcolor}
    \usepackage{pdfpages}
    \usepackage{hyperref}
    \usepackage{listings}
    \usepackage{fancybox}
    \usepackage{lstautogobble}
    \usepackage{titling}
    \usepackage{pdflscape}
    \usepackage[english]{babel}
    \usepackage{comment}
    \usepackage[
    backend=biber,
    style=ieee,
    sorting=ynt
    ]{biblatex}
    \addbibresource{report.bib}
    \usepackage[nottoc,notlot,notlof]{tocbibind}
    \renewcommand\maketitlehooka{\null\mbox{}\vfill}
    \renewcommand\maketitlehookd{\vfill\null}

    \graphicspath{ {Images/} }

    \setlength{\parskip}{\baselineskip}

    \title{Secure Coding Practices:
    \protect\\How modern languages excel, fail and a 
    comparative view of mature languages to modern languages in security.}
    \author{
        Jay Rovacsek
        \texttt{c3146220@uon.edu.au}\\
    }
    \date{\today}
    \hypersetup{
    colorlinks=true,
    linkcolor=black,
    filecolor=magenta,      
    urlcolor=blue,
    citecolor=red,
    linktoc=section,
    }
    \pagenumbering{arabic}

    \lstset{frame=tb,
        language=c++,
        aboveskip=2mm,
        belowskip=2mm,
        showstringspaces=false,
        columns=flexible,
        basicstyle={\small\ttfamily},
        numbers=none,
        numberstyle=\tiny\color{gray},
        keywordstyle=\color{blue},
        commentstyle=\color{dkgreen},
        stringstyle=\color{mauve},
        breaklines=true,
        breakatwhitespace=true,
        tabsize=3
    }

    \begin{document}

    \begin{titlingpage}
        \maketitle
    \end{titlingpage}
    \newpage

    \tableofcontents

    \newpage

    \section{Preface}
    \subsection{Running with Scissors}
        Admittedly, the title\cite{Seacord} for this section is very much thanks to one of the first
        items I read while creating this document. The analogy for development in terms of security 
        could not be more apt for a large portion of the development community.

        Why cover this topic? As a security enthusiast and now developer, I often found myself 
        looking at a system left untouched until absolutely required, the design choices, logic and
        knowledge of the language it was written in left with the author. Commonly a requirement 
        for a hotfix was/is needed in a number of this circumstances as a number of critical 
        business services and resources may rely on the system in question.

        For clarity, I should note that most of these systems only date to the VB6/VB.NET era, suggesting
        and age of roughly 15-20 years at oldest. I look forward to the day of encountering some
        fortran, cobol, pascal or c into the future.
        Knowing some of the most critical systems in the world run on these languages as a base
        gives cause for concern, but is our concern well founded and have modern languages 
        developed enough to allow us to be comfortable into the future?

    \subsection{Tripping Over}
        We can certainly critique early languages for the level of access to the machine they allow 
        a user, without careful consideration in design and a well founded knowledge in the language 
        used issues notorious of early languages. However, in this day and age of highly abstracted 
        languages and frameworks have we traded old demons for new, or do we really have more safety
        in our computing goals?

    \newpage

    \section{The Most Dangerous Component of Computing}
    The apparent and dangerous component of computing is never the computer. Users are terrible,
    will break anything and everything possible and every human is a user. Seasoned veterans of
    technology or never used a piece of electronics, every human is amazingly fallible in comparison 
    to the machine they attempt to drive.

    To avoid heading down the tangent of human computer interaction intricacies, the coverage of
    human issues in modern and mature languages must be covered.

    \newpage

    \section{Modern Language Issues}

    \newpage
    \section{Mature Language Issues}

    \newpage
    \printbibliography

    \end{document}
