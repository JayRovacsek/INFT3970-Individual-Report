\documentclass[12pt]{article}

\pagestyle{empty}

\usepackage[paper=a4paper]{geometry}
\usepackage{flowfram,lipsum}

% \newflowframe[<page list>]{<width>}{<height>}{<x>}{<y>}[<label>]
\newflowframe{10cm}{8cm}{4cm}{16cm}% Top of page
\newflowframe{10cm}{8cm}{4cm}{7.5cm}% Middle of page
\newflowframe{10cm}{8cm}{4cm}{-1cm}% Bottom of page

\setallflowframes{border=plain}% Add frames to each flow frame

\begin{document}
\section{What are these boards?}
The boards used in this project are the ESP12-E.
\\
Just a quick show of hands, who here owns a Raspberry Pi?
\\
Who has used or toyed with a Raspberry Pi before?
\\
For this project we considered using the Pi as our board of choice,
it has wireless, can be powered from a battery and has a small form factor.
\par\noindent
But I think you'll agree that the ESP has a much smaller formfactor just by seeing it, not only that but the board requires much less power and 
we did encounter instances in which utilising some smart programming 
some people could get this board to run for over a month on a single coin 
cell battery.
\par\noindent
The ESP boards come in at around a 12th the price of a Raspberry Pi, at 
\$4 - 5 per board, a 10th the processing power at a whopping 160Mz 
and include a total space onboard to flash of 4MB at most.
\\
Comparitively though, that's around a 20th or less of what a modern x86 
chipset will be expected to run at, the storage about a half millionth 
smaller than the storage we have on modern computerns and the memory for 
operating the chipset? 32KB of SRAM, 80KB of DRAM.
\\
\par\noindent
Issues we encoutnered with this board were only on or two major issues, 
writing of efficient code was super important, without a bit of thought 
that 100KB or so of RAM, we could easily be looking at a memory dump over 
serial to determine what went wrong. 
\\
\par\noindent
Small issues around the specification of the wireless were present, we 
needed to ensure a 2.4Ghz network was reachable, as the boards don't support 
newer 5Ghz ranges of wireless yet. It should be noted that newer models however 
support both zigbee and bluetooth.
\\
\par\noindent
Small break.
\\
\par\noindent
Another show of hands, who here can write arduino code? 
\\
What about c++?
\\
How about any Golang?
\\
How about any Rust?
\\
What about Javascript? Anyone written some of that terrible language?
\\
These boards support all of those languages, and the community is adding more 
reguarly. The barriers to using tech like this is extremely low and only 
requires a small bit of learning to understand the sensors you are using.
\\
\par\noindent
Alternate SBCs or Single Board Computers exist that we could have used, the 
Raspberry Pi Zero or a number of replica style boards, but the ESP offered 
a large number of users already on the platform, the lowest price to start 
without this turning into a pure hardware project and were extremely easy to get
on mass from China.
\\
\par\noindent
Given more time on the project, we wanted to look at extending out more 
metrics we could analyse, namely carbon dioxide detectors, ultrasonic 
senors to generate a proto-mapping of the room and potentially look at 
the noise generated in a room as further metrics.
\\
\par\noindent
We intend on open-sourcing the code to aid other makers into the future and 
give back to the community we learnt so much from over the course of this project.
\\
\par\noindent
I'll now pass along to Josh for a walkthrough of the web application we developed 
to facilitate our analytics of data.
\end{document}